\documentclass{article}

\usepackage[margin=1in]{geometry}
\usepackage{lipsum}
\usepackage{pbox}

\newcommand\headline[3]{
  {\noindent
    \pbox[b]{\textwidth}{#1}\hfill%
    \pbox[b]{\textwidth}{#2}\hfill%
    \pbox[b]{\textwidth}{#3}%
  }%
}

\begin{document}

\headline{
  Sam McHugh \\
  Eli Orvis \\
  Nick Spinale
}{
  {\Large Final Project} \smallskip \\
  \centering {\large Phase 1}
}{
  \hfill CS 257 \\
  \hfill \today
}

\medskip \hrule \vspace{1pt} \hrule height .8pt \medskip
\bigskip

We will make a probabilistic variant of Mine Sweeper.
Instead of containing either a bomb or no bomb,
each square on the board will contain a probability of being a bomb.
Squares which have been uncovered (and were not bombs)
will not contain the number of adjacent bombs,
but either the sum of the probabilities of adjacent squares,
or the actual real-time probability of any of its adjacent squares being a bomb
(which one we choose will depend on which one is more fun).
Controls available to the user before the game starts will include a board size and difficulty selector.
Controls during the game will include a finished button, tools for marking squares with flags,
and a threshold for auto-mine
(squares with probabilities under this threshold will be automatically mined
when one of their neighbors is mined, similarly to how squares with no
adjacent mines are treated in lame Mine Sweeper).
Additionally, the interface will include a clock and statistics such as the total risk taken.
Because it is theoretically possible to clear the entire board,
he player can end the game at any time.
There will then be two final scores:
the number of covered squares remaining (fewer is beter),
and the probability that one of the covered squares is not a bomb (lower is better).

Our application has a well-defined notion of state:
layout of the mine field (including which squares have been uncovered), flags placed, running statistics, user-selected threshold, etc.
The view simply renders this state to the UI:
each uncovered square is rendered with the appropriate label,
and each covered square is rendered with the appropriate marker.
Given these two facts, our planned implementation of probabilistic Mine Sweeper fits the MVC approach well.

\end{document}

